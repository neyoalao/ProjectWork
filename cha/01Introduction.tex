\chapter{Introduction}
\label{cha:Introduction}
\pagenumbering{arabic}

% In the following I will explain in detail the introduction to the topic. In Section \ref{sec:Motivation} I will give a motivation on the thesis topic. Section \ref{sec:Goals} gives an overview of the goals of this thesis and in Section \ref{sec:Overview} I will give an overview of my thesis. 

% \section{Motivation}
% \label{sec:Motivation}
 Agriculture performs a crucial role in the human existence, food security, and economic infrastructure of a country \cite{pawlak2020role}. However, research has shown that the agricultural sector requires the most considerable amount of land and water than any other sector, and it also poses a substantial threat to the biological variety and variability of life on earth \cite{agriculture2008state,green2005farming}. This, amongst many other reasons, has called for the necessity to practice precision farming by monitoring and supplying plants with the correct and precise amount of nutrient or disease control needed for a plentiful and sustainable harvest through technological advancements. Sugar beet - an important biennial cultivated near the temperate region - and sugar cane - grown near the tropical region - are the two vital plants that supply production plants the raw materials needed for the commercial production of sugar in the world \cite{Barreto2020HyperspectralIO, draycott2008sugar}.

 However, Cercospora leaf spot (CLS), beet yellow virus, Ramularia leaf spot, Rhizoctonia root and crown rot, and powdery mildew are some of the significant leaf and root diseases that inhibit the growth of sugar beet plants during cultivation \cite{ozguven2019automatic, Barreto2020HyperspectralIO}. All these diseases reduce the quantity and quality of the potential harvest, decreasing the worth of sucrose extracts. Conventional methods of disease detection and classification are through visual inspections by experts and laboratory tests of plant leaves or soil samples\cite{yang2018classification}. However, they have proven to be quite tedious, not necessarily effective, and time-consuming amongst several limitations \cite{lu2021review}. Hence a need for automation of the detection of diseases in plants using technologies like machine learning and UAV to optimise the performance of this process.
As an advancement, machine learning methods have been used in different studies to classify plant diseases using features like texture, type and colour of plant leaf images. Examples of machine learning methods used are K-nearest neighbours (KNN), random forest (RF), and Support vector machine (SVM) \cite{Barreto2020HyperspectralIO,sujatha2021performance}. Nevertheless, models trained using machine learning techniques are inefficient/impractical in performing real-time classification of plant leaf diseases. 
The lack of diversity in the quality of plant leaf image dataset and the careful need for segmentation of features of interest from raw data are some of the significant reasons that account for the inefficiencies of the trained machine learning models \cite{arivazhagan2013detection, athanikar2016potato}.

The shortcomings of these machine learning techniques motivated current research on using deep learning (DL) techniques. Convolutional neural network (CNN) - a type of DL network - has shown promising results for use in plant disease classification \cite{kawasaki2015basic}. In research papers, the CNN network has proven to be efficient for plant disease detection and classification, thanks to their strong automatic feature extraction networks \cite{ma2018recognition, ferentinos2018deep}. With DL, plant disease detection is possible since there are significant changes in healthy and infected plant leaves, and the pixel-wise operations of CNN layers can pick up these changes on the leaf images \cite{lu2021review}.

In conclusion, with the recent advancements in research and GPU development, deep learning might be a probable solution to conventional disease monitoring and detection methods. Furthermore, models can be trained and deployed to tools like mobile phones, drones, and websites.

 This paper briefly introduces the economic importance of sugar beets, common diseases that affect them, and limitations of existing conventional disease detection methods, followed by an overview of proposed techniques in research papers. The three common diseases that affect sugar beets are described in Chapter \ref{cha:Fundamentals}.  This includes Cercospora leaf spot (CLS), Powdery mildew and Rhizoctonia Root and Crown Rot (RCRR) diseases. The disease assessment sensors (Hyperspectral imaging and RGB imaging) for capturing leaf information and examples of studies that used these sensors are introduced in Chapter \ref{cha:acessDisease}. Chapter \ref{cha:Analysis} analyses studies on machine learning and deep learning approaches for disease detection in sugar beets. 

% \section{Goals}
% \label{sec:Goals}
% G O A L S.

% \section{Overview}
% \label{sec:Overview}
% The required fundamentals of this thesis are explained in Chapter \ref{cha:Fundamentals}. These includes the fundamental basics of automata theory. Related work is discussed in Chapter \ref{cha:RelatedWork}. The detailed consideration of the influences on this work, such as existing systems to be considered, user requirements and environmental influences are analyzed in Chapter \ref{cha:Analysis}. Based on this analysis as well as related work, a design is presented in Chapter \ref{cha:DesignImplementation} that considers the posed requirements of this work.  An evaluation of the developed approach is presented in Chapter 6, where the requirements set are reflected on. Finally, a summary and outlook are presented in Chapter \ref{cha:SummaryOutlook}.  





