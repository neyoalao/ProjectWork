\chapter{Conclusion}\label{cha:conclusion}
This paper reviewed studies done in the area of non-invasive leaf disease monitoring and detection in sugar beets. First, the paper discussed the causative agents, signs and symptoms of three common diseases that affect the quantity and quality of yields in a sugar beet plantation season in Chapter \ref{cha:Fundamentals}. Second, Chapter \ref{cha:acessDisease} of the paper presented the hyperspectral and the RGB imaging sensors for capturing information from plant leaves in section \ref{rgb} and \ref{hyperspectral}. Furthermore, the sections explored different researches that used hyperspectral imaging for leaf disease detection tasks. Third, the paper gave an introduction to machine learning and CNN in Chapter \ref{cha:Analysis}. Then the remaining sections examined different research papers that conducted studies using machine learning and CNN algorithms by citing their techniques for data capturing, pre-processing, training algorithms and accuracies.

While there are researches in the direction of using machine learning algorithms for leaf disease detection and monitoring in sugar beets, the feature extraction from datasets are dependent on human intervention, as evidenced in the studies reviewed in this paper. However, there are currently not enough researches exploring a deep learning-based approach for leaf disease detection in sugar beet. Therefore, more research needs to be done in this area since CNN-based architectures can automate the feature extraction process in a given dataset, reducing the human intervention needed. Likewise, more research needs to be carried out using datasets captured in field environments under different lighting and atmospheric conditions since the trained models are to be used in field environments, not in controlled laboratory settings common in research papers.

